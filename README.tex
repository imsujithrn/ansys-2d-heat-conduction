\documentclass{article}
\usepackage{amsmath}
\usepackage{graphicx}
\usepackage{hyperref}
\usepackage{enumitem}

\begin{document}

\title{2D Heat Conduction Problem - ANSYS Simulation}
\author{Your Name}
\date{\today}
\maketitle

\section*{Overview}
This repository contains an ANSYS Workbench simulation solving a steady-state 2D heat conduction problem with mixed boundary conditions. The simulation focuses on determining the \textbf{dimensionless temperature distribution} and \textbf{heat flux} in a rectangular domain subjected to different thermal boundary conditions.

\section*{Problem Description}
\textbf{Geometry:} A rectangular domain with width \( W \) and height \( H = 2W \).

\textbf{Boundary Conditions:}
\begin{itemize}
    \item Bottom wall: Isothermal boundary \( \theta = 1 \) (constant temperature).
    \item Top and left walls: Adiabatic boundary \( \frac{\partial \theta}{\partial y} = 0 \) and \( \frac{\partial \theta}{\partial x} = 0 \), respectively (no heat flux).
    \item Right wall: Convective boundary \( -\frac{\partial \theta}{\partial x} = \text{Bi} \theta \), with a Biot number \( \text{Bi} = 5 \).
\end{itemize}

\textbf{Aspect Ratio:} \( H/W = 2 \).

\textbf{Solver:} Steady-state thermal analysis in ANSYS Mechanical.

\textbf{Objective:} To compute the dimensionless temperature distribution and heat flux.

\section*{Repository Structure}
\begin{verbatim}
/
|-- geometry/          # Contains CAD files (if any)
|-- meshes/            # Contains mesh files or settings
|-- results/           # Stores screenshots or plots of the results
|-- macros/            # Any ANSYS macros used for automation
|-- project_files/     # ANSYS project files (e.g., .wbpj, .agdb, etc.)
|-- docs/              # Documentation, screenshots, or relevant papers
|-- .gitignore         # Specifies which files to ignore in version control
|-- LICENSE            # License under which this project is distributed
|-- README.md          # This project overview and setup instructions
\end{verbatim}

\section*{How to Use}
\begin{enumerate}[label=\arabic*.]
    \item \textbf{Clone the repository:}
    \begin{verbatim}
    git clone https://github.com/yourusername/ansys-2d-heat-conduction.git
    \end{verbatim}
    \item \textbf{Open the ANSYS project:}
    Navigate to the \texttt{project\_files/} directory and open the project file (\texttt{heat\_conduction\_project.wbpj}) in ANSYS Workbench.
    \item \textbf{Run the simulation:}
    Ensure that the geometry, mesh, and boundary conditions are correctly set up as described in the problem statement. Run the steady-state thermal analysis to compute the results.
    \item \textbf{Post-Processing:}
    Analyze the dimensionless temperature distribution and heat flux using ANSYS Mechanical post-processing tools. Result images can be found in the \texttt{results/} folder or generated using ANSYS.
\end{enumerate}

\section*{Dependencies}
\begin{itemize}
    \item \textbf{ANSYS Workbench:} This project was developed and tested on ANSYS Workbench [Version X.X] (replace with your version).
    \item \textbf{No additional libraries:} The project requires only ANSYS Workbench to run.
\end{itemize}

\section*{Results}
The simulation results include:
\begin{itemize}
    \item \textbf{Temperature Distribution:} The dimensionless temperature field in the rectangular domain.
    \item \textbf{Heat Flux:} The distribution of heat flux across the domain boundaries.
\end{itemize}
You can visualize the results after solving the model in ANSYS. Screenshots of sample results are included in the \texttt{docs/} or \texttt{results/} directory.

\section*{License}
This project is licensed under the \textbf{MIT License}. See the \texttt{LICENSE} file for more details.

\section*{Contributing}
Contributions are welcome! Please submit issues or pull requests if you'd like to propose any changes or improvements to the project.

\section*{Acknowledgments}
This project was created as part of an academic simulation on steady-state 2D heat conduction. Special thanks to [add contributors or collaborators if applicable].

\end{document}
